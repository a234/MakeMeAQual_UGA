%%%%%%%%%%%%%%%%%%%%%%%%%%%%%%%%%%%%%%%%%%%%%%%%%%%%%%%%%%%%%%%%%%%%%%%%%%%%%%%%
%%%%%%%%%%%%%%%%%%%%%%%%%%%%%%%%%%%%%%%%%%%%%%%%%%%%%%%%%%%%%%%%%%%%%%%%%%%%%%%%

% PACKAGES %

\documentclass[11pt]{amsart}
\usepackage[main=english]{babel}
\usepackage[utf8]{inputenc}
\usepackage{amssymb}
\usepackage{amsmath}
\usepackage{amsthm}
\usepackage{enumerate, enumitem}
\usepackage{colonequals}

%%%%%%%%%%%%%%%%%%%%%%%%%%%%%%%%%%%%%%%%%%%%%%%%%%%%%%%%%%%%%%%%%%%%%%%%%%%%%%%%
%%%%%%%%%%%%%%%%%%%%%%%%%%%%%%%%%%%%%%%%%%%%%%%%%%%%%%%%%%%%%%%%%%%%%%%%%%%%%%%%

% THEOREMS %

\theoremstyle{plain}
\newtheorem{theorem}{Theorem}
\newtheorem{corollary}{Corollary}[theorem]
\newtheorem{lemma}{Lemma}
\newtheorem{proposition}{Proposition}
\newtheorem{conjecture}{Conjecture}

\theoremstyle{definition}
\newtheorem{definition}{Definition}
\newtheorem{example}{Example}
\newtheorem{observation}[theorem]{Observation}

\theoremstyle{remark}
\newtheorem{remark}[theorem]{Remark}
\newtheorem{question}{Question}
\newtheorem{problem}{Problem}

%%%%%%%%%%%%%%%%%%%%%%%%%%%%%%%%%%%%%%%%%%%%%%%%%%%%%%%%%%%%%%%%%%%%%%%%%%%%%%%%
%%%%%%%%%%%%%%%%%%%%%%%%%%%%%%%%%%%%%%%%%%%%%%%%%%%%%%%%%%%%%%%%%%%%%%%%%%%%%%%%

% MATH OPERATORS %

\DeclareMathOperator{\Z}{\mathbb{Z}}
\DeclareMathOperator{\N}{\mathbb{N}}
\DeclareMathOperator{\R}{\mathbb{R}}
\DeclareMathOperator{\F}{\mathbb{F}}
\DeclareMathOperator{\Q}{\mathbb{Q}}
\DeclareMathOperator{\C}{\mathbb{C}}

\newcommand{\mf}[1]{\mathfrak{#1}}
\newcommand{\mc}[1]{\mathcal{#1}}

\setlength\parindent{0pt}

%%%%%%%%%%%%%%%%%%%%%%%%%%%%%%%%%%%%%%%%%%%%%%%%%%%%%%%%%%%%%%%%%%%%%%%%%%%%%%%%
%%%%%%%%%%%%%%%%%%%%%%%%%%%%%%%%%%%%%%%%%%%%%%%%%%%%%%%%%%%%%%%%%%%%%%%%%%%%%%%%

% COMMENTS %
\newcommand{\sk}[1]{\textcolor{BurntOrange}{\textbf{[#1]}}}


%%%%%%%%%%%%%%%%%%%%%%%%%%%%%%%%%%%%%%%%%%%%%%%%%%%%%%%%%%%%%%%%%%%%%%%%%%%%%%%%
%%%%%%%%%%%%%%%%%%%%%%%%%%%%%%%%%%%%%%%%%%%%%%%%%%%%%%%%%%%%%%%%%%%%%%%%%%%%%%%%

% TITLE %

\begin{document}

\title{Some Quals Problems}
\author{Santiago Arango-Piñeros}
\date{\today}

\maketitle

%%%%%%%%%%%%%%%%%%%%%%%%%%%%%%%%%%%%%%%%%%%%%%%%%%%%%%%%%%%%%%%%%%%%%%%%%%%%%%%%
%%%%%%%%%%%%%%%%%%%%%%%%%%%%%%%%%%%%%%%%%%%%%%%%%%%%%%%%%%%%%%%%%%%%%%%%%%%%%%%%


% BODY %
\section{Algebra}
\subsection{Groups}
    \begin{enumerate}
        \item Classify the groups of order $182 = 2 \cdot 7 \cdot 13$.
        
        \item  Let $G$ be a finite group of order $p^nm$ where $p$ is a prime and $m$ is not divisible by $p$. Prove that if $H$ is a subgroup of $G$ of order $p^k$ for some $k<n$, then the normalizer of $H$ in $G$ properly contains $H$.

        
        \item Let $H$ be a subgroup of $S_n$ of index $n$. Prove:
        \begin{enumerate}
            \item[a)] There is an isomorphism $f: S_n \to S_n$ such that $f(H)$ is the subgroup of $S_n$ stabilizing $n$. In particular, $H$ is isomorphic to $S_{n-1}$.
            \item[b)] The only subgroups of $S_n$ containing $H$ are $S_n$ and $H$.
        \end{enumerate}
        
        \item \begin{itemize}
        \item[a)] Prove that a group of order $351=3^3\cdot 13$ cannot be simple.
        \item[b)] Prove that a group of order $33$ must be cyclic.
    \end{itemize}
        
        \item \begin{enumerate}
        \item[a)] Let $G$ be a group, and $Z(G)$ the center of $G$. Prove that if $G/Z(G)$ is cyclic, then $G$ is abelian.
        \item[b)] Prove that a group of order $p^n$, where $p$ is a prime and $n \geq 1$, has non-trivial center.
        \item[c)] Prove that a group of order $p^2$ must be abelian.
    \end{enumerate}
    
    \item Let $G$ be a finite group.
    \begin{enumerate}
        \item[a)] Prove that if $H < G$ is a proper subgroup, then $G$ is not the union of conjugates of $H$.
        \item[b)] Suppose that $G$ acts transitively on a set $X$ with $|X| > 1$. Prove that there exists an element of $G$ with no fixed points in $X$.
    \end{enumerate}
    
    \item Classify all groups of order $15$ and of order $30$.
        
    \item Count the number of $p$-Sylow subgroups of $S_p$.
    
    \item \begin{enumerate}
        \item[a)] Let $G$ be a group of order $n$. Suppose that for every divisor $d$ of $n$, $G$ contains at most one subgroup of order $d$. Show that $G$ is clyclic.
        \item[b)] Let $F$ be a field. Show that every finite subgroup of the group of units $F^\times$ is cyclic.
    \end{enumerate}
    \end{enumerate}
\subsection{Fields and Galois Theory}

    \begin{enumerate}
        \item Let $K$ and $L$ be finite fields. Show that $K$ is contained in $L$ if and only if $\# K = p^r$ and $\# L = p^s$ for the same prime $p$, and $r \leq s$.
        
        \item Let $K$ and $L$ be finite fields with $K \subseteq L$. Prove that $L$ is Galois over $K$ and that $\mathrm{Gal}(L/K)$ is cyclic.
        
        \item Fix a field $F$, a separable polynomial $f\in F[x]$ of degree $n \geq 3$, and a splitting field $L$ for $f$. Prove that if $[L:F] = n!$ then:
    \begin{enumerate}
        \item[a)] $f$ is irreducible.
        \item[b)] For each root $r$ of $f$, $r$ is the unique root of $f$ in $F(r)$.
        \item[c)] For every root $r$ of $f$, there are no proper intermediate fields $F \subset L \subset F(r)$.
    \end{enumerate}
    
    \item \begin{enumerate}
        \item[a)] Show that $\sqrt{2+\sqrt{2}}$ is a root of $p(x) = x^2 - 4x^2 + 2 \in \mathbf{Q}[x]$.
        \item[b)] Prove that $\mathbf{Q}(\sqrt{2 + \sqrt{2}})$ is a Galois extension of $\mathbf{Q}$ and find its Galois group. (Hint: note that $\sqrt{2 - \sqrt{2}}$ is another root of $p(x)$).
        \item[c)] Let $f(x) = x^3 - 5$. Determine the splitting field $K$ of $f(x)$ over $\mathbf{Q}$ and the Galois group of $f(x)$. Give an example of a proper sub-extension $\mathbf{Q} \subset L \subset K$, such that $L/\mathbf{Q}$ is Galois.
    \end{enumerate}
    \end{enumerate}

\subsection{Rings}

    \begin{enumerate}
        \item An integral domain $R$ is said to be an {\it Euclidean domain} if there is a function $N: R \to \{n\in\mathbf{Z} \mid n\geq 0\}$ such that $N(0)=0$ and for each $a,b\in R$ with $b\neq 0$, there exist elements $q,r\in R$ with
    \begin{align*}
        a = qb + r, \quad \text{and} \quad r = 0 \, \text{ or } \, N(r) < N(b).
    \end{align*}
    Prove:
    \begin{enumerate}
        \item[a)] The ring $F[[x]]$ of power series over a field $F$ is an Euclidean domain.
        
        \item[b)] Every Euclidean domain is a PID. 
    \end{enumerate}
    
    \item Let $F$ be a field, and let $R$ be the subring of $F[X]$ of polynomials with $X$ coefficient equal to $0$. Prove that $R$ is not a UFD.
    
    \item $R$ is a commutative ring with 1. Prove that if $I$ is a maximal ideal in $R$, then $R/I$ is a field. Prove that if $R$ is a PID, then every nonzero prime ideal in $R$ is maximal. Conclude that if $R$ is a PID and $p\in R$ is prime, then $R/(p)$ is a field.
    
    \end{enumerate}

\subsection{Linear Algebra}

    \begin{enumerate}
        \item Prove that any square matrix is conjugate to its transpose matrix. (You may prove it over $\mathbf{C}$).
        
        \item Determine the number of conjugacy classes of $16 \times 16$ matrices with entries in $\mathbf{Q}$ and minimal polynomial $(x^2+1)^2(x^3+2)^2$.
        
        \item Let $V$ be a vector space over a field $F$. The evaluation map $e\colon V \to (V^\vee)^\vee$ is defined by $e(v)(f) \colonequals f(v)$ for $v\in V$ and $f\in V^\vee$.
            \begin{enumerate}
                \item[a)] Prove that $e$ is an injection.
                \item[b)] Prove that $e$ is an isomorphism if and only if $V$ is finite dimensional.
            \end{enumerate}
            
        \item Let $R$ be a principal ideal domain that is not a field, and write $F$ for its field of fractions. Prove that $F$ is not a finitely generated $R$-module.
        
        \item Carefully state Zorn's lemma and use it to prove that every vector space has a basis.
    \end{enumerate}

\section{Analysis}
\subsection{Complex Analysis}
\begin{enumerate}
    \item Use residues to compute the integral
    \begin{align*}
        \int_{0}^{\infty} \dfrac{\cos x}{(x^2+1)^2} \mathrm{d}x \, .
    \end{align*}
    
    \item State and prove the Cauchy integral formula for holomorphic functions.
    
    \item Use the Cauchy integral formula to prove the maximal principle for analytic functions.
    
    \item Let $f$ be an entire function and suppose that $|f(z)| \leq A|z|^2$ for all $z$ and some constant $A$. Show that $f$ is a polynomial of degree $\leq 2$.
    
    \item \begin{enumerate}
        \item[a)] State the Schwarz lemma for analytic functions in the unit disc.
        
        \item[b)] Let $f: \mathbf{D} \to \mathbf{D}$ be an analytic map from the unit disc $\mathbf{D}$ into itself. Use the Schwarz lemma to show that for each $a\in \mathbf{D}$ we have
     \begin{align*}
         \dfrac{|f'(a)|}{1-|f(a)|^2} \leq \dfrac{1}{1-|a|^2} \, .
     \end{align*}
    \end{enumerate}
    
    \item State the Riemann mapping theorem and prove the uniqueness part.
    
    \item Compute the integrals
    \begin{align*}
        \int_{|z-2|=1} \dfrac{e^z}{z(z-1)^2} \, 
        \mathrm{d}z, \quad \int_0^\infty \dfrac{\cos 2x}{x^2 + 2} \, \mathrm{d}x \, .
    \end{align*}
    
    \item Let $(f_n)$ be a sequence of holomorphic functions in a domain $D$. Suppose that $f_n \to f$ uniformly on each compact subset of $D$. Show that
    \begin{itemize}
        \item[a)] $f$ is holomorphic on $D$.
        
        \item[b)] $f_n' \to f'$ uniformly on each compact subset of $D$.
    \end{itemize}
    
    \item If $f$ is a non-constant entire function, then $f(\mathbf{C})$ is dense in the plane.
    
    \item \begin{enumerate}
        \item[a)] State Rouche's theorem.
        \item[b)] Let $f$ be analytic in a neighborhood of $0$, and satisfying $f'(0) \neq 0$. Use Rouche's theorem to show that there exists a neighborhood $U$ of $0$ such that $f$ is a bijection in $U$.
    \end{enumerate}
    
    \item Let $f$ be a meromorphic function in the plane such that
   \begin{align*}
       \lim_{|z|\to\infty} |f(z)| = \infty \, .
   \end{align*}
   \begin{enumerate}
       \item[a)] Show that $f$ has only finitely many poles.
       \item[b)] Show that $f$ is a rational function.
   \end{enumerate}
\end{enumerate}

\subsection{Real Analysis}
\begin{enumerate}
    \item Describe the process that extends a measure on an algebra $\mathcal{A}$ of subsets of $X$, to a complete measure defined on a $\sigma$-algebra $\mathcal{B}$ containing $\mathcal{A}$. State the corresponding definitions and results (without proofs).
    
    \item State and prove Fatou's Lemma on a general measurable space.
    
    \item \begin{enumerate}
        \item[a)] State the Dominated Convergence Theorem for Lebesgue integrals.
        
        \item[b)] Let $\{f_n\}$ be a sequence of measurable functions on a Lebesgue measurable set $E$ which converges {\it in measure} to a function $f$ on $E$. Suppose that for every $n$, $|f_n| \leq g$ with $g$ integrable on $E$. Using the above theorem show that 
        \begin{align*}
            \int_E |f_n-f| \longrightarrow 0 \, .
        \end{align*}
    \end{enumerate}
    
    \item Let $f\in L^1([0,1])$. Show that
    \begin{enumerate}
        \item[a)] The limit $\lim_{p\to 0^+} \| f \|_p$ exists.
        \item[b)] If $m \{x : f(x) = 0\} > 0$, then the above limit is zero.
    \end{enumerate}
    
    \item Let $f$ be a continuous function on $[0,1]$. Show that the following statements are equivalent.
    \begin{enumerate}
        \item[a)] $f$ is absolutely continuous.
        \item[b)] For any $\epsilon > 0$ there exists $\delta > 0$ such that $m(f(E)) < \epsilon$ for any set $E\subseteq [0,1]$ with $m(E) < \delta$.
        \item[c)] $m(f(E)) = 0$ for any set $E \subseteq [0,1]$ with $m(E)=0$.
    \end{enumerate}
\end{enumerate}








%%%%%%%%%%%%%%%%%%%%%%%%%%%%%%%%%%%%%%%%%%%%%%%%%%%%%%%%%%%%%%%%%%%%%%%%%%%%%%%%
%%%%%%%%%%%%%%%%%%%%%%%%%%%%%%%%%%%%%%%%%%%%%%%%%%%%%%%%%%%%%%%%%%%%%%%%%%%%%%%%



\end{document}


